\documentclass[12pt]{article}
 
\usepackage[margin=1in]{geometry} 
\usepackage{amsmath,amsthm,amssymb,bm}
\usepackage{graphicx}
 
\begin{document}
  
\title{Task 4. Cloth Rendering}
\author{Garoe Dorta-Perez\\
CM50245: Computer Animation and Games II}
 
\maketitle
 
\section{Introduction}

Rendering realistic images is a challenging task, specially if there are memory or time constrains for the computation.
Cloth is a complex material composed of interwoven threads of different types.
Moreover, its appearance can vary from diffuse to highly specular.

\section{Previous work}

Several methods have been proposed to render cloth fabrics efficiently and realistically.
One of the earliest approaches was based on simple empirical shading models \cite{Weft1986}.
The main objective was to accomplish believe shading, disregarding physical accuracy.
As a general division, there are image based approached, geometric models and volumetric models.

\bibliographystyle{plain}
\bibliography{t4_report}

\end{document}

